\newcommand{\taems}{\acro{t\ae ms}}

\chapter{Coordination Using Goal and Plan Hierarchies}
\label{cha:hier-coord}


Goal, plan, and policy hierarchies have proven to be very successful
methods for coordinating agents. In these approaches we assume the
existance of one of these hierarchies and the problem then becomes
that of determining which parts of the hierarchy are to be done by
which agents. In this setting we assume that the agents are
coopertive, that is, they will do exactly what we tell them to. The
problem is one of finding an good-enough answer. 

Hierarchies like the ones we show here are a way to solve the problem
of finding an optimal policy in multiagent MDPs. That is, since the
traditional mathematical tools for solving those problems are
computationally intractable for real-world problems, researchers
hoping to build real-world systems have had to find ways to add more
of their domain knowledge into the problem description so as to make
it easier to find a solution. In general, the more we tell the agent
about the world the easier it is to solve the problems it faces.
Unfortunately, if we want to tell the agent more information about its
domain we need more sophisticated data structures. In this chapter we
describe one of the most popular such data structures: \taems{}. Once
we have these new data structures we will generally need new
algorithms which take advantage of the new information. The algorithms
used for \taems{} are captured in the \acro{gpgp} framework which we also
discuss.

\section{\taems{}}
\label{sec:taems}

\td{\taems{}} is a language for representing large task hierarchies
that contain complex constraints among tasks. You can think of it as a
data structure for representing very complicated constraint
optimization problems \cite{lesser04a}.  At its simplest, \taems{}
represents a goal hierarchy. As such, \taems{} structures are roughly
tree-shaped.  The root is the top-level goals that we want the system
to achieve.  The children are the sub-goals that need to be achieved
in order for the top goal to be achieved---a form of divide and
conquer. The leafs of the tree are either goals that can be achieved
by a single agent or tasks that can be done by an agent. These goals
and tasks, however, might require the use of some resources or data.
\taems{} represents these requirements with an arrow from the data or
resource to the goal.


%These techniques were developed by Victor Lesser group at U.
%Massachusetts at Amherst over the last couple of decades.


\begin{SCfigure}
  \begin{minipage}{1.0\linewidth}
    \begin{center}
      \begin{tikzpicture}[line width=1pt]
        \node [draw,circle] (g0) at (0,0) {$G_0$};
        \node [draw,circle] (g1) at (-3,-2) {$G_1$};
        \node [draw,circle] (g2) at (0,-2) {$G_2$};
        \node [draw,circle] (g3) at (3,-2) {$G_3$};
        \node [draw,circle] (g21) at (-2,-4) {$G_{21}$};
        \node [draw,circle] (g22) at (0,-4) {$G_{22}$};
        \node [draw,circle] (g23) at (2,-4) {$G_{23}$};
        \node [draw,circle] (g31) at (4,-4) {$G_{31}$};
        \draw (g0) -- (g1) node[coordinate,near start] (c1) {}
        (g0) -- (g2)
        (g0) -- (g3) node[coordinate,near start] (c2) {}
        (g2) -- (g21) node[coordinate,near start] (c21) {}
        (g2) -- (g22)
        (g3) -- (g23) node[coordinate,near start] (c31) {}
        (g2) -- (g23) node[coordinate,near start] (c22) {}
        (g3) -- (g31) node[coordinate,near start] (c32) {};
        \draw[color=gray] (c1) .. controls +(0,-.5) and +(0,-.5)
        .. node[below,near end,gray] {\textcolor{black}{and}} (c2);
        \draw[color=gray] (c21) .. controls +(0,-.4) and +(0,-.4)
        .. node[below,near end,gray] {\textcolor{black}{and}} (c22);
        \draw[color=gray] (c31) .. controls +(0,-.3) and +(0,-.3)
        .. node[below,gray] {\textcolor{black}{or}} (c32);
        \node (d1) at (-3,-6) {$data_1$};
        \node (d2) at (-1,-6) {$data_2$};
        \node (r1) at (1,-6) {$resource_1$};
        \node (r2) at (3,-6) {$resource_2$};
        \draw (g1) -- (d1)
        (g21) -- (d1)
        (g22) -- (d1)
        (g22) -- (r1)
        (g23) -- (d2)
        (g31) -- (r2);
        \draw[-angle 60,gray] (g1) -- node[sloped,above,near start] {\textcolor{black}{enables}}
        (g22);
        \draw[-angle 60,gray] (g2) -- node[sloped,above] {\textcolor{black}{enables}} (g3);
        \node[anchor=south] at (-3,-1.5) {
          \begin{small}
            \setlength{\tabcolsep}{0pt}
            \begin{tabular}{rl}
              quality\textbf{:} & (.2,0)(.8,8) \\
              cost\textbf{:} & (1,0) \\
              duration\textbf{:} & (1,2)
            \end{tabular}
          \end{small}
        };
        \node[anchor=north west] at (4,-4.4) {
          \begin{small}
            \setlength{\tabcolsep}{0pt}
            \begin{tabular}{rl}
              q\textbf{:} & (.1,0)(.9,5) \\
              c\textbf{:} & (1,10) \\
              d\textbf{:} & (.4,2)(.6,5)
            \end{tabular}
          \end{small}
        };
      \end{tikzpicture}
    \end{center}
  \end{minipage}
  \caption{An example of a simple \taems{} structure.}
  \label{fig:taemsbasic}
\end{SCfigure}


Figure~\ref{fig:taemsbasic} shows a simple example \taems{} structure.
It tells us that in order to achieve goal $G_0$ we must do $G_1$ and
$G_2$ and $G_3$. But, in order to achieve $G_3$ we must achieve either
$G_{23}$ or $G_{31}$. Note also how $G_{23}$ is a subgoal of both
$G_2$ and $G_3$, which makes this a graph instead of a tree. The
bottom row shows two data elements and two resources. Data elements
are pieces of data that are needed in order to achieve the goals to
which they are connected. Resources are consumable resources that are
needed by their particular goals. The difference between data and
resource is that a piece of data can be re-used as many times as
needed while resources are consumed every time they are used and must
therefore be replenished at some point. \taems{} also allows one to
annotate links to resources to show how much of the resource is needed
in order to achieve a particular goal. We can see that $G_1$ requires
$data_1$ while $G_{22}$ requires $data_1$ and $resource_1$.

Figure~\ref{fig:taemsbasic} also shows \td{non local effects},
namely the \id{enables} constraint. The directed link (in red) from
$G_1$ to $G_{22}$ indicates that $G_1$ enables $G_{22}$ which means
that $G_1$ must be achieved before we can start to work on $G_{22}$ if
we want $G_{22}$ to be achieved with quality.

The \taems{} formal model calls for each leaf goal or task to be
described by three parameters: quality, cost, and duration.
Figure~\ref{fig:taemsbasic} shows examples of these parameter values
for only two nodes ($G_1$ and $G_{31}$) due to space constraints.  In
practice all leaf nodes should have similar values. These parameters
represent the quality we expect to get when the goal is achieved (that
is, how well it will get done), the cost the system will incur in
achieving that goal, and the duration of time required to achieve the
goal (or perform the task). The values are probabilistic. For example,
$G_{31}$ will with .1 probability be achieved with a quality of 0
(that is, it will fail to get achieved) while with a probability of .9
it will be achieved with quality of 5. $G_{31}$ also has a fixed cost
of 10 and it will take either 2 time units to finish, with probability
.4, or 5 units to finish, with probability .6.

Figure~\ref{fig:taemsbasic} shows two ways of aggregating the quality
of children nodes: the and and or functions. These are boolean
functions and require us to establish a quality threshold below which
we say that a goal has not been achieve. \taems{} usually assumes that
a quality of 0 means that the task was not achieved and anything
higher means that the task was achieved. For example, $G_0$ will only
be achieved if (because of the \id{and}) $G_1$ is achieved, which
only happens with probability .8 each time an agent tries.


\begin{SCfigure}
  \begin{minipage}{1.0\linewidth}
  \begin{center}
  \begin{tabular}{rl} \toprule
    Function & Description \\ \midrule
    $q_{min}$ & minimum quality of all subtasks \\
    $q_{max}$ & maximum quality of all subtasks \\
    $q_{sum}$ & aggregate quality of all subtasks \\
    $q_{last}$ & quality of most recently completed subtask \\
    $q_{sum\_all}$ & as with $q_{sum}$ but all subtasks must be
    completed \\
    $q_{seq\_min}$ & as with $q_{min}$ but all subtasks must be
    completed in order \\
    $q_{seq\_max}$ & as with $q_{max}$ but all subtasks must be
    completed in order \\ \bottomrule
  \end{tabular}
  \end{center}
  \end{minipage}
  \caption{Some of the quality accumulation functions available for a
    \taems{} structure.}
  \label{tab:qafs}
\end{SCfigure}

\taems{} allows us to use many different functions besides simply
\id{and} and \id{or}. Table~\ref{tab:qafs} shows some of the
\td{quality accumulation functions} (qafs) that are defined in
\taems{}. Of course, the more different qafs we use the harder it
will be, in general, to find a good schedule for the agents.
Scheduling algorithms must always limit themselves to a subset of
qafs in order to achieve realistic running times.

\section{GPGP}

Generalized Partial Global Planning (\acro{gpgp}) is the complete framework
for multiagent coordination which includes, as one of its parts the
\taems{} data structure. Over the years there have been many \acro{gpgp}
agent implementations each one slightly different from the others but
all sharing some basic features.  They have been applied to a wide
variety of applications, from airplane repair scheduling, to supply
chain management, to distributed sensor networks. The \taems{} data
structure and the \acro{gpgp} approach to coordination via hierarchical task
decomposition are among the most successful and influential ideas in
multiagent systems.

\subsection{Agent Architecture}

\begin{SCfigure} 
  \begin{minipage}{1.0\linewidth}
    \begin{center} 
      \begin{tikzpicture}[line width=1pt]
        \node[draw,rounded corners] (taems) at (0,0) {
          \begin{tabular}{c}
            \taems{} Structure \\ and \\ Goal Criteria
          \end{tabular}};
        \node[draw] (dtc) at (-4.5,3) {
          \begin{tabular}{c}
            Design-to-Criteria \\ Scheduler
          \end{tabular}};
        \node[draw] (execution) at (4,3) {
          \begin{tabular}{c}
            Execution
          \end{tabular}};
        \node[draw] (task) at (4,0) {
          \begin{tabular}{c}
            Task \\ Assessor
          \end{tabular}};
        \node[draw] (gpgp) at (0,-3) {
          \begin{tabular}{c}
            \acro{gpgp} \\ Coordination
          \end{tabular}};
        \node[draw,rounded corners] (commitment) at (-4.5,-2) {
          \begin{tabular}{c}
            Non-Local \\ Commitment \\ Database
          \end{tabular}};
        \node[draw,rounded corners] (schedule) at (0,3) {
          \begin{tabular}{c}
            Schedule
          \end{tabular}};
        \draw[-angle 60] (taems) -- node[above,sloped] {Uses}
        (dtc);
        \draw[-angle 60] (commitment) -- node[above,sloped] {Uses}
        (dtc);
        \draw[-angle 60] (dtc) -- node[above] {Produces}
        (schedule);
        \draw[-angle 60] (schedule) -- node[above] {Uses}
        (execution);
        \draw[angle 60-angle 60] (execution.south east) -- +(0,-4) node[left]
        {\begin{tabular}{c}
            Action/Sense \\ Domain Info. Msgs.
          \end{tabular}};
        \draw[-angle 60] (execution.south west) 
        .. controls +(-.5,-.5) and +(.5,-.5) .. 
        node[below] {Reschedule Requests} (dtc.south east);
        \draw[-angle 60] (execution) -- node[above,sloped] {State}
        (task);
        \draw[-angle 60] (execution) -- node[below,sloped] {Updates}
        (taems);
        \draw[-angle 60] (task) -- node[above,sloped] {Updates}
        (taems);
        \draw[-angle 60] (gpgp) -- node[below,sloped] 
        {Reschedule Requests} (dtc.south);
        \draw[-angle 60] (gpgp) -- node[above,sloped] {Updates}
        (commitment);
        \draw[-angle 60] (gpgp) -- node[above,sloped] {Updates}
        (taems);
        \draw[angle 60-angle 60] (gpgp) -- +(2,.5) node[right]
        {Coordination Msgs};
      \end{tikzpicture}
    \end{center}
  \end{minipage}
  \caption{\acro{gpgp} agent architecture. Components are in squares and data
    structures are in rounded squares.}
  \label{fig:gpgpagent}
\end{SCfigure}


Figure~\ref{fig:gpgpagent} shows a typical agent architecture using
\acro{gpgp}. The red squares are software components. These typically operate
in parallel within the agent. The black rounded squares are some of
the most important data structures used by a \acro{gpgp} agent.

The \taems{} data structure lies at the heart of a \acro{gpgp} agent. This
\taems{} structure is also annotated with the goal criteria. That is,
it specifies which goals the agent should try to achieve. Remember
that the basic \taems{} structure only lists all possible goals, it
does not say which agent should try to achieve which goal. The \taems{}
structure used by an agent does contain information that tells the
agent who is trying to achieve which task. The agent gets this
information from the \acro{gpgp} coordination module which we will explain
later.

The design to criteria scheduler is a component (program) which takes
as input a \taems{} data structure, along with some commitments the
agent has made, and generates a schedule. We will discuss the
commitments in more detail later but, basically, the agent has the
ability to make commitments to other agents that say that it will
achieve a certain goal by a certain time. Once an agent has made a
commitment the design to criteria scheduler does everything possible
to ensure that the schedule it produces satisfies that commitment.
The schedules produced by the scheduler are simply lists of tasks or
goals the agent must execute and the times at which the agent will
execute them.

The execution module handles plan execution as well as monitoring.
That is, it uses the schedule generated by the scheduler and instructs
the agent to perform the next task according to the schedule. It then
uses the agent's sensors to make sure that the task was accomplished.
In the case of a failure, or if it notices that the next task/goal is
impossible to achieve in the current world state, it will ask the
design to criteria scheduler to generate a new schedule. The execution
module also sends all the information it has about the world and the
schedule's execution state to the task assessor module. The task
assessor module has domain-specific knowledge and can determine if
there are changes that need to be made to the \taems{} structure. For
example, it might determine that one of the subgoals is no longer
needed because of some specific change in the world. The execution
module can also, in a more generalized manner, make changes to the
\taems{} structure. These changes are all incorporated into the
current \taems{} structure.

\subsection{Coordination}

\begin{SCfigure}
  \begin{minipage}{1.0\linewidth}
    \begin{center}
      \begin{tikzpicture}[line width=1pt]
        \node [draw,circle] (g0) at (-5,0) {$G_0^{1}$};
        \node [draw,circle] (g1) at (-6,-2) {$G_1^{1,2}$};
        \node [draw,circle] (g2) at (-4,-2) {$G_2^{2}$};
        \draw (g0) -- (g1) (g0) -- (g2);
        \draw[-angle 60] (g1) -- (g2);

        \node [draw,circle] (g0b) at (0,0) {$G_0^{1}$};
        \node [draw,circle] (g1b) at (-1,-2) {$G_1^{*}$};
        \node [draw,circle] (g3b) at (-2,-4) {$G_3^{1}$};
        \node [draw,circle] (g4b) at (0,-4) {$G_4^{2}$};
        \node [draw,circle] (g2b) at (1,-2) {$G_2^{2}$};
        \draw (g0b) -- (g1b) (g0b) -- (g2b) 
        (g1b) -- (g3b) node[coordinate,near start] (c1b) {}
        (g1b) -- (g4b) node[coordinate,near start] (c2b) {};
        \draw (c1b) .. controls +(0,-.2) and +(0,-.2)
        .. node[below] {max} (c2b);
        \draw[-angle 60] (g1b) -- (g2b);
      \end{tikzpicture}

      \bigskip

      \begin{tikzpicture}[line width=1pt]
        \node [draw,circle] (g01) at (0,0) {$G_0^{1}$};
        \node [draw,circle] (g11) at (-1,-2) {$G_1^{1}$};
        \node [draw,circle] (g31) at (-2,-4) {$G_3^{1}$};
        \draw (g01) -- (g11)
        (g11) -- (g31) node[coordinate,near start] (c1) {};

        \node [draw,circle] (g02) at (3,0) {$G_0^{1}$};
        \node [draw,circle] (g12) at (2,-2) {$G_1^{2}$};
        \node [draw,circle] (g42) at (3,-4) {$G_4^{2}$};
        \node [draw,circle] (g22) at (4,-2) {$G_2^{2}$};
        \draw (g02) -- (g12) (g02) -- (g22) 
        (g12) -- (g42) node[coordinate,near start] (c2) {};
        \draw[-angle 60] (g12) -- (g22);
        \draw[dashed,-angle 60] (g11)
        .. controls +(1,1) and +(-1,1) .. node[above] {CR} (g22);
        \draw[dashed,-angle 60] (g31) -- node[above] {CR} (g12);
        \draw[dashed,-angle 60] (g42) -- node[above] {CR} (g11);
      \end{tikzpicture}
    \end{center}
  \end{minipage}
  \caption{Example of how \acro{gpgp} forms coordination relationships.
    Superscripts on the goals indicate the number of the agents that
    can achieve that goal. The original \taems{} structure, on the top
    left, is expanded, top right, and two graphs generated from it and
    given to agents 1 and 2, bottom. The coordination relationships
    are then added.}
\label{fig:coord}
\end{SCfigure}

In parallel with all these modules, the \acro{gpgp} coordination module is
always in communication with other agents and tries to keep the
agent's commitments and \taems{} structure updated. The \acro{gpgp}
coordination module implements (mostly) domain independent
coordination techniques and is the module that tells the agent which
of the goals and tasks in its \taems{} structure it should achieve.

Figure~\ref{fig:coord} gives a pictorial view of the general steps the
\acro{gpgp} module follows in order to achieve coordination among agents. The
coordination module starts with the \taems{} structure shown on the
left. The structure is then modified so that every goal that can be
achieved by more than one agent is replaced by a new dummy goal which
has one child for every agent that could perform the goal. Each one of
the children is a new goal that can only be performed by one child.
The children are joined by a $\max$ qaf. In Figure~\ref{fig:coord} we
see show $G_1^{1,2}$, which could be achieved by either 1 or 2 (as
denoted by the superscript) is replaced by a new $G_1$ with two
children $G_3^1$ and $G_4^2$.

The resulting graph is then divided among the agents such that each
agent has the subgraph with all the goals that the agent can achieve
plus all the ancestors of these goals (all the way to the root). This
division can be seen in the last two graphs of Figure~\ref{fig:coord}.

The \acro{gpgp} coordination module identifies \td{coordination
  relationships} (CR). These come from two different situations: when
a non-local effect in the original graph now starts in one graph and
ends in another, or when a non-local effect or a subtask relationship
has one end in one subgraph but the other end in both subgraphs.

Notice that these CRs denote situations where one of the agents needs
to either wait for the other to finish or to determine whether the
other one has failed. The \acro{gpgp} module uses some mechanism to remove
the uncertainty that these CRs cause. Specifically, it makes the agent
commit to do those actions that are at the initiating end of the CR to
the other agent. For example, in Figure~\ref{fig:coord} the \acro{gpgp}
module would make agent 1 send to agent 2 the messages: ``Commit
(\textbf{Do}($G_1$))'' and ``Commit (\textbf{Do}($G_3$))''.


\subsection{Design-to-Criteria Scheduler}

The design-to-criteria scheduler is a complex piece of software which
uses search along with a bunch of heuristics to find a good schedule
given the current \taems{} structure and the set of commitments. The
scheduler only tries to schedule the first few tasks since it knows
that it is likely that the problem will change and it will need to
re-schedule. The scheduler also sometimes needs to create temporary
commitments and generate schedules using those commitments for the
\acro{gpgp} coordination module. That is, when the \acro{gpgp} coordination module
wants to know how a certain commitment might affect the agent's other
commitments it will ask the design-to-criteria scheduler to create a
new schedule using those commitments. 

The job of the scheduler is very complicated. It must decide how to
satisfy goals in the \taems() structure and which ones are likely to
generate the maximum utility. As such, the scheduler is a large piece
of code which is generally considered a black box by \acro{gpgp} users. They
just feed it the \taems{} structure and out comes a schedule. The
scheduler solves this problem by using a number of search heuristics
and other specialized techniques.

\subsection{GPGP/\taems{} Summary}

The \acro{gpgp}/\taems{} framework utilizes some key concepts. It views the
problem of coordination as distributed optimization, in fact, it
reduces complex problems to something similar to the distributed
constraint optimization problem of Chapter~\ref{ch:dconstraints}. It
provides a family of coordination mechanism for situation-specific
control. It uses \taems{} to provide proven domain-independent
representation of agent tasks, with support for multiple goals of
varying worth and different deadline. It also uses quantitative
coordination relationships among tasks.

The general approach of \acro{gpgp} has been widely successful and has
proven itself in many domains. The \taems{} structure has been adopted
not just as a programming tools but also as a design tool to be used
to better understand the problem. However, one problem with \acro{gpgp} has
been that it has required modifications for each one of its
applications. As such, \acro{gpgp} is best thought of as a general approach
to designing multiagent systems rather than as a specific
tool/architecture.

%%% Local Variables: 
%%% mode: latex
%%% TeX-command-default: "PDFlatex"
%%% TeX-master: "~/wp/mas/mas"
%%% End: 


